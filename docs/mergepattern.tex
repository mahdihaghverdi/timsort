\section{توضیح الگوریتم - ادغام‌ها}
\begin{frame}{بهتر کردن استفاده از فضا در ادغام}
\begin{itemize}\itemr
\item[-]
الگوریتم ادغام در \lr{Merge sort} الگوریتمی \lr{inplace} نیست و برای ادغام کردن دو آرایه با طول‌های
\m{n}
و
\m{m}
یک آرایه‌ی جدید به طول
\m{n + m}
نیاز دارد و در کل برای ادغام ما چنین فضایی را اشغال کرده‌ایم:
\begin{flushleft}
\m{\underbrace{\m{n}}_{len(X)} + \underbrace{\m{m}}_{len(Y)} + \underbrace{\m{n + m}}_{\text{\lr{new array}}}}
\end{flushleft}

\item[-]
اما تیم‌سورت برای \lr{inplace} مرتب کردن و کمتر کردن 
\lr{space overhead}
الگوریتم مرتب سازی ادغامی از روش بهتری برای ادغام کردن دو \lr{Run} استفاده می‌کند.
\end{itemize}
\end{frame}

\begin{frame}{بهتر کردن استفاده از فضا در ادغام (ادامه)}
\begin{itemize}\itemr
\item[-]
اگر آرایه‌ای به این صورت داشته باشیم:
\begin{lfl}
\m{A = [\, \underbrace{\m{12, 19, 21, 22}}_{\m{X}}, \, \underbrace{\m{3, 5, 17, 22, 107, 109}}_{\m{Y}} \,]}
\end{lfl}
دو \lr{Run} عه 
\m{X}
و
\m{Y}
در آن می‌توان شناسایی کرد.

\item[-]
در این روش جدید ادغام ما یک آرایه‌ی موقت به اندازه‌ی \lr{Run} کوچک‌تر درست می‌کنیم و عناصر \lr{Run} کوچک‌تر را در آن کپی می‌کنیم. کپی کردن‌های این چنین هم (یعنی یک تکه از یک آرایه) به بهینه‌ترین حالت در سطح پایین اجرا می‌شوند. پس حالا ما چنین چیزی داریم:

\begin{lfl}
\m{A = [12, 19, 21, 22, 3, 5, 17, 22, 107, 109]}\\
\m{T = [12, 19, 21, 22]}
\end{lfl}
\end{itemize}
\end{frame}

\begin{frame}{بهتر کردن استفاده از فضا در ادغام (ادامه)}
\begin{itemize}\itemr
\item[-]
چون آرایه‌ی کوچک‌تر سمت چپ بود، پس از سمت چپ‌ آرایه‌ی موقت و دومین \lr{Run} شروع به ادغام کردن می‌کنیم. که یعنی:
\begin{lfl}
\m{A = [12, 19, 21, 22, \mathcircled{\m{3}}, 5, 17, 22, 107, 109]}\\
\m{T = [\mathcircled{\m{12}}, 19, 21, 22]}
\end{lfl}
را باهم مقایسه، و برنده (عنصر کوچک‌تر) را در ایندکس صفر آرایه‌ی \m{A} می‌گذاریم. که می‌شود:

\begin{lfl}
\m{A = [\textcolor{red}{\m{3}}, 19, 21, 22, \mathcircled{\m{3}}, 5, 17, 22, 107, 109]}\\
\m{T = [\mathcircled{\m{12}}, 19, 21, 22]}
\end{lfl}
\end{itemize}
\end{frame}

\begin{frame}{بهتر کردن استفاده از فضا در ادغام (ادامه)}
\begin{itemize}\itemr
\item[-]
سپس:
\begin{lfl}
\m{A = [3,
19,
21,
22,
\textcolor{blue}{\mathcircled{\m{3}}},
\mathcircled{\m{5}}, 17, 22, 107, 109]}\\
\m{T = [\mathcircled{\m{12}}, 19, 21, 22]}
\end{lfl}

\item[-]
که می‌شود:
\begin{lfl}
\m{A = [
\textcolor{red}{\m{3}}, 
\textcolor{red}{\m{5}}, 
21, 
22, 
\textcolor{blue}{\mathcircled{\m{3}}}, 
\mathcircled{\m{5}}, 17, 22, 107, 109]}\\
\m{T = [\mathcircled{\m{12}}, 19, 21, 22]}
\end{lfl}
\end{itemize}
\end{frame}

\begin{frame}{بهتر کردن استفاده از فضا در ادغام (ادامه)}
\begin{itemize}\itemr
\item[-]
سپس:
\begin{lfl}
\m{A = [3, 
19, 
21, 
22, 
\textcolor{blue}{\mathcircled{\m{3}}}, 
\textcolor{blue}{\mathcircled{\m{5}}}, 
\mathcircled{\m{17}}, 22, 107, 109]}\\
\m{T = [\mathcircled{\m{12}}, 19, 21, 22]}
\end{lfl}

\item[-]
که می‌شود:
\begin{lfl}
\m{A = [
\textcolor{red}{\m{3}}, 
\textcolor{red}{\m{5}}, 
\textcolor{red}{\m{12}}, 
22, 
\textcolor{blue}{\mathcircled{\m{3}}}, 
\textcolor{blue}{\mathcircled{\m{5}}}, 
\mathcircled{\m{17}}, 22, 107, 109]}\\
\m{T = [
\mathcircled{\m{12}}, 
19, 
21, 
22]}
\end{lfl}
\end{itemize}
\end{frame}
