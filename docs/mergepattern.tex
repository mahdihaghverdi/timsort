\section{توضیح الگوریتم - ادغام‌ها}
\begin{frame}{بهتر کردن استفاده از فضا در ادغام}
\begin{itemize}\itemr
\item[-]
الگوریتم ادغام در \lr{Merge sort} الگوریتمی \lr{inplace} نیست و برای ادغام کردن دو آرایه با طول‌های
\m{n}
و
\m{m}
یک آرایه‌ی جدید به طول
\m{n + m}
نیاز دارد،
\item[-]
اما تیم‌سورت برای \lr{inplace} مرتب کردن و کمتر کردن 
\lr{space overhead}
الگوریتم مرتب سازی ادغامی از روش بهتری برای ادغام کردن دو \lr{Run} استفاده می‌کند.
\end{itemize}
\end{frame}

\begin{frame}{بهتر کردن استفاده از فضا در ادغام (ادامه)}
\begin{itemize}\itemr
\item[-]
اگر آرایه‌ای به این صورت داشته باشیم:
\begin{lfl}
\m{A = [\, \underbrace{\m{12, 19, 21, 22}}_{\m{X}}, \, \underbrace{\m{3, 5, 17, 22, 107, 109}}_{\m{Y}} \,]}
\end{lfl}
دو \lr{Run} عه 
\m{X}
و
\m{Y}
در آن می‌توان شناسایی کرد.

\item[-]
در این روش جدید ادغام ما یک آرایه‌ی موقت به اندازه‌ی \lr{Run} کوچک‌تر درست می‌کنیم و عناصر \lr{Run} کوچک‌تر را در آن کپی می‌کنیم. کپی کردن‌های این چنین هم (یعنی یک تکه از یک آرایه) به بهینه‌ترین حالت در سطح پایین اجرا می‌شوند. پس حالا ما چنین چیزی داریم:

\begin{lfl}
\m{A = [\,\underbrace{\m{12, 19, 21, 22}}_{\m{X
}}, \underbrace{\m{3, 5, 17, 22, 107, 109}}_{\m{Y}}\,]}\\
\m{T = [12, 19, 21, 22]}
\end{lfl}
\end{itemize}
\end{frame}

\begin{frame}{گام‌ها}
\begin{itemize}\itemr
\item[-]
چون آرایه‌ی کوچک‌تر سمت چپ بود، پس از سمت چپ‌ آرایه‌ی موقت و دومین \lr{Run} شروع به ادغام کردن می‌کنیم. که یعنی:
\begin{lfl}
\m{A = [12, 19, 21, 22, \mathcircled{\m{3}}, 5, 17, 22, 107, 109]}\\
\m{T = [\mathcircled{\m{12}}, 19, 21, 22]}
\end{lfl}
را باهم مقایسه، و برنده (عنصر کوچک‌تر) را در ایندکس صفر آرایه‌ی \m{A} می‌گذاریم. که می‌شود:

\begin{lfl}
\m{A = [\textcolor{red}{\m{3}}, 19, 21, 22, \mathcircled{\m{3}}, 5, 17, 22, 107, 109]}\\
\m{T = [\mathcircled{\m{12}}, 19, 21, 22]}
\end{lfl}
\end{itemize}
\end{frame}

\begin{frame}{گام‌ها (ادامه)}
\begin{itemize}\itemr
\item[-]
سپس:
\begin{lfl}
\m{A = [
\textcolor{red}{3}, 
19, 
21, 
22, 
\textcolor{blue}{\m{3}}, 
\mathcircled{\m{5}}, 17, 22, 107, 109]}\\
\m{T = [\mathcircled{\m{12}}, 19, 21, 22]}
\end{lfl}

\item[-]
که می‌شود:
\begin{lfl}
\m{A = [
\textcolor{red}{\m{3}}, 
\textcolor{red}{\m{5}}, 
21, 
22, 
\textcolor{blue}{\m{3}}, 
\mathcircled{\m{5}}, 17, 22, 107, 109]}\\
\m{T = [\mathcircled{\m{12}}, 19, 21, 22]}
\end{lfl}
\end{itemize}
\end{frame}

\begin{frame}{گام‌ها (ادامه)}
\begin{itemize}\itemr
\item[-]
سپس:
\begin{lfl}
\m{A = [
\textcolor{red}{3}, 
\textcolor{red}{5},
21, 
22, 
\textcolor{blue}{\m{3}}, 
\textcolor{blue}{\m{5}},
\mathcircled{\m{17}}, 22, 107, 109]}\\
\m{T = [\mathcircled{\m{12}}, 19, 21, 22]}
\end{lfl}

\item[-]
که می‌شود:
\begin{lfl}
\m{A = [
\textcolor{red}{\m{3}}, 
\textcolor{red}{\m{5}}, 
\textcolor{red}{\m{12}}, 
22, 
\textcolor{blue}{\m{3}}, 
\textcolor{blue}{\m{5}}, 
\mathcircled{\m{17}}, 22, 107, 109]}\\
\m{T = [
\mathcircled{\m{12}}, 
19, 
21, 
22]}
\end{lfl}
\end{itemize}
\end{frame}

\begin{frame}{گام‌ها (ادامه)}
\begin{itemize}\itemr
\item[-]
سپس:
\begin{lfl}
\m{A = [
\textcolor{red}{\m{3}}, 
\textcolor{red}{\m{5}}, 
\textcolor{red}{\m{12}}, 
22, 
\textcolor{blue}{\m{3}}, 
\textcolor{blue}{\m{5}}, 
\mathcircled{\m{17}}, 22, 107, 109]}\\
\m{T = [
\textcolor{blue}{\m{12}}, 
\mathcircled{\m{19}}, 
21, 
22]}
\end{lfl}

\item[-]
که می‌شود:
\begin{lfl}
\m{A = [
\textcolor{red}{\m{3}}, 
\textcolor{red}{\m{5}}, 
\textcolor{red}{\m{12}}, 
\textcolor{red}{\m{17}}, 
\textcolor{blue}{\m{3}}, 
\textcolor{blue}{\m{5}}, 
\mathcircled{\m{17}}, 22, 107, 109]}\\
\m{T = [
\textcolor{blue}{\m{12}}, 
\mathcircled{\m{19}}, 
21, 
22]}
\end{lfl}
\end{itemize}
\end{frame}

\begin{frame}{گام‌ها (ادامه)}
\begin{itemize}\itemr
\item[-]
سپس:
\begin{lfl}
\m{A = [
\textcolor{red}{\m{3}}, 
\textcolor{red}{\m{5}}, 
\textcolor{red}{\m{12}}, 
\textcolor{red}{\m{17}}, 
\textcolor{blue}{\m{3}}, 
\textcolor{blue}{\m{5}}, 
\textcolor{blue}{\m{17}},
\mathcircled{\m{22}},
107,
109]}\\
\m{T = [
\textcolor{blue}{\m{12}}, 
\mathcircled{\m{19}}, 
21, 
22]}
\end{lfl}

\item[-]
که می‌شود:
\begin{lfl}
\m{A = [
\textcolor{red}{\m{3}}, 
\textcolor{red}{\m{5}}, 
\textcolor{red}{\m{12}}, 
\textcolor{red}{\m{17}}, 
\textcolor{red}{\m{19}}, 
\textcolor{blue}{\m{5}}, 
\textcolor{blue}{\m{17}},
\mathcircled{\m{22}}, 22, 107, 109]}\\
\m{T = [
\textcolor{blue}{\m{12}}, 
\mathcircled{\m{19}}, 
21, 
22]}
\end{lfl}
\end{itemize}
\end{frame}

\begin{frame}{گام‌ها (ادامه)}
\begin{itemize}\itemr
\item[-]
سپس:
\begin{lfl}
\m{A = [
\textcolor{red}{\m{3}}, 
\textcolor{red}{\m{5}}, 
\textcolor{red}{\m{12}}, 
\textcolor{red}{\m{17}}, 
\textcolor{red}{\m{19}}, 
\textcolor{blue}{\m{5}}, 
\textcolor{blue}{\m{17}}, 
\mathcircled{\m{22}}, 22, 107, 109]}\\
\m{T = [
\textcolor{blue}{\m{12}}, 
\textcolor{blue}{\m{19}}, 
\mathcircled{\m{21}}, 
22]}
\end{lfl}

\item[-]
که می‌شود:
\begin{lfl}
\m{A = [
\textcolor{red}{\m{3}}, 
\textcolor{red}{\m{5}}, 
\textcolor{red}{\m{12}}, 
\textcolor{red}{\m{17}}, 
\textcolor{red}{\m{19}}, 
\textcolor{red}{\m{21}}, 
\textcolor{blue}{\m{17}}, 
\mathcircled{\m{22}}, 107, 109]}\\
\m{T = [
\textcolor{blue}{\m{12}}, 
\textcolor{blue}{\m{19}}, 
\mathcircled{\m{21}},
22]}
\end{lfl}
\end{itemize}
\end{frame}

\begin{frame}{گام‌ها (ادامه)}
\begin{itemize}\itemr
\item[-]
سپس:
\begin{lfl}
\m{A = [
\textcolor{red}{\m{3}}, 
\textcolor{red}{\m{5}}, 
\textcolor{red}{\m{12}}, 
\textcolor{red}{\m{17}}, 
\textcolor{red}{\m{19}}, 
\textcolor{red}{\m{21}}, 
\textcolor{blue}{\m{17}},
\mathcircled{\m{22}},
107,
109]}\\
\m{T = [
\textcolor{blue}{\m{12}}, 
\textcolor{blue}{\m{19}}, 
\textcolor{blue}{\m{21}},
\mathcircled{\m{22}}]
}
\end{lfl}

\item[-]
که در اینجا عنصر ۲۲ آرایه‌ی موقت انتخاب می‌شود که برای حفظ پایداری الگوریتم مرتب‌ سازی لازم است که بیست و دویی که زودتر آمده را انتخاب کند:
\begin{lfl}
\m{A = [
\textcolor{red}{\m{3}}, 
\textcolor{red}{\m{5}}, 
\textcolor{red}{\m{12}}, 
\textcolor{red}{\m{17}}, 
\textcolor{red}{\m{19}}, 
\textcolor{red}{\m{21}}, 
\textcolor{red}{\m{22}},
22,
107,
109]}\\
\m{T = [
\textcolor{blue}{\m{12}}, 
\textcolor{blue}{\m{19}}, 
\textcolor{blue}{\m{21}},
\mathcircled{\m{22}}]}
\end{lfl}
\end{itemize}
\end{frame}

\begin{frame}{گام‌ها (ادامه)}
\begin{itemize}\itemr
\item[-]
وضعیت کنونی ادغام:
\begin{lfl}
\m{A = [
\textcolor{red}{\m{3}}, 
\textcolor{red}{\m{5}}, 
\textcolor{red}{\m{12}}, 
\textcolor{red}{\m{17}}, 
\textcolor{red}{\m{19}}, 
\textcolor{red}{\m{21}}, 
\textcolor{red}{\m{22}}, 
\mathcircled{\m{22}}, 22, 107, 109]}\\
\m{T = [
\textcolor{blue}{\m{12}}, 
\textcolor{blue}{\m{19}}, 
\textcolor{blue}{\m{21}}, 
\textcolor{blue}{\m{22}}]}
\end{lfl}
\item[-]
و حالا که عناصر آرایه‌ی موقت تمام شده، باقی عناصر باقی مانده سر جای خود هستند و دیگر مقایسه و جابجایی لازم نیست و آرایه به صورت مرتب ادغام شد:

\begin{lfl}
\m{A = [
\textcolor{red}{\m{3}}, 
\textcolor{red}{\m{5}}, 
\textcolor{red}{\m{12}}, 
\textcolor{red}{\m{17}}, 
\textcolor{red}{\m{19}}, 
\textcolor{red}{\m{21}}, 
\textcolor{red}{\m{22}},
\textcolor{red}{\m{22}},
\textcolor{red}{\m{107}},
\textcolor{red}{\m{109}}]}
\end{lfl}
\end{itemize}
\end{frame}

\begin{frame}{جهت ادغام}
\begin{itemize}\itemr
\item[-]
در مثال قبلی \lr{Run} کوچک‌تر در سمت چپ بود و ما ادغام را از سمت چپ \lr{Run} بزرگ‌تر و آرایه‌ی موقت انجام دادیم، اما اگر چنین حالتی باشد:
\begin{lfl}
\m{A = [\, \underbrace{\m{3, 5, 17, 22, 107, 109}}_{\m{X}}, \, \underbrace{\m{12, 19, 21, 22}}_{\m{Y}} \,]}
\end{lfl}

\item[-]
ما چنین چیزی داریم:
\begin{lfl}
\m{A = [\,\underbrace{\m{3, 5, 17, 22, 107, 109}}_{\m{X}}, \underbrace{\m{12, 19, 21, 22}}_{\m{Y}}\,]}\\
\m{T = [12, 19, 21, 22]}
\end{lfl}
\end{itemize}
\end{frame}

\begin{frame}{جهت ادغام (ادامه)}
\begin{itemize}\itemr
\item[-]
در این حالت که 
\lr{Run}
کوچک‌تر در سمت راست بود، ما هم تمام ادغام را از سمت راست انجام می‌دهیم

\begin{lfl}
\m{A = [3, 5, 17, 22, 107, \mathcircled{\m{109}}, 12, 19, 21, 22]}\\
\m{T = [
12,
19,
21,
\mathcircled{\m{22}}]}
\end{lfl}

\item[-]
که می‌شود:
\begin{lfl}
\m{A = [3, 5, 17, 22, 107, \mathcircled{\m{109}}, 12, 19, 21, \textcolor{red}{109}]}\\
\m{T = [
12,
19,
21,
\mathcircled{\m{22}}]}
\end{lfl}
\end{itemize}
\end{frame}

\begin{frame}{جهت ادغام (ادامه)}
\begin{itemize}\itemr
\item[-]
و سپس:
\begin{lfl}
\m{A = [3, 5, 17, 22, \mathcircled{\m{107}}, \textcolor{blue}{109}, 12, 19, 21, \textcolor{red}{109}]}\\
\m{T = [
12,
19,
21,
\mathcircled{\m{22}}]}
\end{lfl}

\item[-]
که می‌شود:
\begin{lfl}
\m{A = [3, 5, 17, 22, \textcolor{blue}{107}, \textcolor{blue}{109}, 12, 19, \textcolor{red}{107}, \textcolor{red}{109}]}\\
\m{T = [
12,
19,
21,
\mathcircled{\m{22}}]}
\end{lfl}
و الی آخر...

\item[-]
تابعی که عمل ادغام را شروع می‌کند، تابع
\lr{\texttt{merge\_collapse()}}\fn{1}{\url{https://github.com/python/cpython/blob/3.10/Objects/listobject.c\#L1956}}
است که در ادامه بیشتر راجع به آن توضیح داده خواهد شد.
\end{itemize}
\end{frame}
