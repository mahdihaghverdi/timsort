\section{ادغام پیشرفته‌تر با \lr{Galloping}}
\begin{frame}{تکنیک \lr{Galloping}}
\begin{itemize}\itemr
\item[-]
چنین سناریویی را تصور کنید: ما این آرایه‌ را داریم:

\begin{lfl}
\m{A = [\, \underbrace{\m{1, 2, 3, 6, 10}}_{\m{X}}, \, \underbrace{\m{4, 5, 7, 9, 12, 14, 17}}_{\m{Y}} \,] }
\end{lfl}

\item[-]
در تکنیک \lr{Galloping}، ما با استفاده از \lr{binary search} مکان اولین عضو 
\m{Y}
را در 
\m{X}
پیدا می‌کنیم که می‌بینیم در جایگاه ۴ باید آن را بگذاریم، \textbf{این یعنی تمامی عناصر \m{[1, 2, 3]} در جایگاه درستی قرار دارند}.

\item[-]
همین کار را برای آخرین عنصر 
\m{X}
نسبت به 
\m{Y}
انجام می‌دهیم که مکان آن در جایگاه پنجم دومین \lr{Run} است، \textbf{این یعنی تمامی عناصر \m{12, 14, 17} در جایگاه درستی قرار دارند}.

\item[-]
پس ما چنین آرایه‌ای داریم:
\begin{lfl}
\m{A = [\textcolor{red}{\m{1, 2, 3}}, \, \underbrace{\m{6, 10}}_{\m{X}}, \, \underbrace{\m{5, 7, 9}}_{\m{Y}}, \textcolor{red}{\m{12, 14, 17}}]}
\end{lfl}
و صرفا باید الگوریتم ادغامی که بالاتر بحث شد را روی این 
\m{X}
و 
\m{Y}
کوچک‌تر انجام بدهیم که هم سریع‌تر است و هم فضای کمتری استفاده خواهد کرد.
\end{itemize}
\end{frame}
